\chapter{Instal·lació de software}\label{ch:software-installation}

El procés de desenvolupament s’ha dut a terme emprant un servidor preparat pels directors del projecte, amb les següents característiques:

\begin{itemize}
    \item \textbf{Processador:}
    \begin{itemize}
        \item quatre cores
    \end{itemize}
    \item \textbf{Emmagatzemament:}
    \begin{itemize}
        \item 16 GB de disc de sistema
        \item 256 GB de disc SSD per dades
    \end{itemize}
\end{itemize}

\noindent
Aquest disposa de dos usuaris, tfe, per l’ús diari, i un altre, amb privilegis d’administrador, root, en cas que es necessitin aquests.
L’accés es realitza a través de l’usuari tfe, utilitzant una connexió ssh:

\begin{verbatim}
    $ ssh tfe@ostia.epsevg.upc.edu
\end{verbatim}

\noindent
Les tasques principals realitzades són:

\begin{itemize}
    \item Desenvolupament del codi d’OSTIA (Open Science Toolkit Information Access), utilitzant el sistema de control de versions git.
    \item Anàlisis, dels logs d’UPCommons, amb registres des del 2006 fins al 2023.
    \item Processament i abocament dels logs a la base de dades InfluxDb, ubicada al mateix servidor.
    \item Descàrrega de totes les metadades del servidor OAI sbivdev1.
    \item Anàlisis, processament i abocament de les metadades a una base de dades MongoDb.
    \item Anàlisi quantitativa i qualitatiu del conjunt de les dades, amb el suport l’eina Grafana.
\end{itemize}

