\chapter{Open Science Toolkit Information Access}\label{ch:ostia}
El repositori principal on es troba el codi emprat per tractar i analitzar les dades d'accés a \gls{UPCommons} s'anomena Open Science Toolkit Information Access (\gls{OSTIA}).

\noindent \\
En aquest s'hi ha desenvolupat les eines necessàries per la realització del projecte.
Es tracta d'un repositori basat en el sistema de control de versions \textit{\gls{git}}.

\noindent \\
La mecànica de treball sobre aquest tipus de repositoris és la tradicional.
Per cada nova característica que es vulgui afegir al codi font de l'eina, es crea una nova branca.
Tot el procés de desenvolupament es fa sobre aquesta, deixant la branca principal pel codi acabat.

\noindent \\
Per no divergir gaire amb la principal, un cop s'acaba la característica que es vol implementar, es disposa a fer la fusió.
Aquest procés es sol fer mitjançant una petició (al context de git, \textit{Pull Request}), ja qué com a bona pràctica, la branca prinicpal està protegida per tal d'evitar a tota costa la introducció d'errors, \textit{bugs}, etc.

\noindent \\
Aquesta petició per afegir el nou codi sol ser revisada per altres membres de l'equip de desenvolupadors, els contribuidors a l'eina, responsables\dots

\noindent \\
En el nostre cas, la feina ha estat realitzada individualment, per tant no tenim aquests segons punts de vista.
Malgrat això, el procés s'ha dut a terme igualment per dos motius: mantenit un històric de canvis, i modularitzar el desenvolupamnet.

\clearpage

\section*{Estructura}\label{sec:ostia-structure}

\noindent
Dividirem l'estructura del repositori de codi en dues parts, encara que convisquin ambdues:
\begin{itemize}
    \item Codi relacionat amb el correcte funcionament del mateix repositori.
    \item Codi relacionat amb el processament i l'anàlisi de les dades d'accés a \gls{UPCommons}
\end{itemize}

\noindent \\
La primera part està estructurada de la següent manera:

\begin{verbatim}
    - .github/workflows
    - LICENSE
    - README.md
    - Makefile
    - Pipfile
    - pre-commit-config.yaml
\end{verbatim}

\noindent \\
El propòsit individual és:

\begin{itemize}
    \item \texttt{.github/workflow}: Directori on s'inclouen tots els arxius que defineixen els \textit{\gls{GitHub}} \textit{workflows}.
    Parlem d'accions que s'executen puntualment, com per exemple, després d'afegir codi a la branca principal, durant el procés de \textit{Pull Request}, etc.
    \item \texttt{LICENSE}: llicència sota la qual es desenvolupa el programari.
    \item \texttt{README.md}: per norma general, fitxer informatiu principal als repositoris de \textit{\gls{GitHub}}.
    \item \texttt{Makefile}: fitxer d'automatització de comandes.
    L'utilitzarem per generar la documentació, comprimir el codi principal, crear la imatge de \textit{\gls{Docker}}
    \item \texttt{Pipfile}: fitxer que defineix les dependències, conjuntament amb la versió específica, necessàries pel funcioanment de l'eina.
    \item \texttt{pre-commit-config.yaml}: fitxer de configuració que determina un seguit de normes que s'han de complir abans d'enviar un \textit{\gls{commit}}
\end{itemize}

\clearpage

\noindent \\
Per altra banda, tenim la següent estructura:

\begin{verbatim}
    - config
    - doc
    - lib
    - scripts
    - src
      |-- logs
      |-- metadata
      |-- queries
\end{verbatim}

\noindent \\
El propòsit de cada directori és:

\begin{itemize}
    \item \texttt{config}: Emmagatzema els diferents fitxers de configuració requerits durant el processament de les dades.
    Tant la base de dades com els diferents \textit{\gls{plugin}s} utilitzats es troben definits aquí. \\
    Vegeu configuració del servidor~\ref{sec:server-configuration}
    \item \texttt{doc}: Documentació del codi.
    Classes, mètodes, comentaris, excepcions, etc.
    Inclou TODO! una eina per generar manualment aquesta documentació
    \item \texttt{lib}: Dependències necessàries pel funcionament de l'eina.
    \begin{tcolorbox}[colback=blue!5!white, colframe=blue!75!black, title=Interoperabilitat]
        Com menys dependències externes tingui un programari, aquest serà més versàtil, operable i multiplataforma.
    \end{tcolorbox}
    \item \texttt{scripts}: Conjunt de \textit{scripts} que ajuden a automatitzar certes tasques.
    No formen part del codi principal de l'eina, però si que en complementen aquesta.
    \item \texttt{src}: Directori principal de codi.
    Es troba el codi font emprat per la realització del projecte.
    Està escrit en Python.
    Diferenciem els tres casos d'ús:
    \begin{itemize}
        \item \texttt{logs}: processament i emmagatzematge dels \textit{logs}.
        \item \texttt{metadata}: processament i emmagatzematge de les metadades.
        \item \texttt{queries}: anàlisi de les dades.
    \end{itemize}
\end{itemize}

