\chapter{Procés d'anonimització dels \textit{\gls{log}s}}\label{ch:log-anonymization}

\noindent
Com els \textit{\gls{log}s} originals contenen adreces \gls{IP} reals que són considerades dades personals, s'ha hagut de passar per un procés d'anonimització per poder fer ús d'aquestes.

\noindent \\
Aquest procés es basa a assignar una nova adreça IP vàlida a partir de la real de forma aleatòria, mantenint la consistència per generar sempre la mateixa parella d'adreces. \\

\begin{tcolorbox}[colback=green!5!white, colframe=green!50!black, title=Comitè d'Ètica de la UPC]
    El comité d'Ètica va emetre un dictamen \textbf{favorable} on considera que les activitats realitzades al projecte respecten els principis del Codi Ètic de la UPC\@.
    \tcblower
    Codi d'identifiació: 2024\texttt{-}007
\end{tcolorbox}

\noindent \\
Podeu trobar informació més detallada al pla de gestió dades, generat amb l'eina DMP (\url{https://dmp.csuc.cat/}), que s'inclou a continuació.

\includepdf[scale=0.8,pages=-]{pdfs/data-management-plan.pdf}
