\chapter{Servidor de treball}\label{ch:server-description}

El procés de desenvolupament s’ha dut a terme emprant un servidor preparat pels directors del projecte, amb les següents característiques:

\begin{itemize}
    \item \textbf{Processador:}
    \begin{itemize}
        \item quatre cores
    \end{itemize}
    \item \textbf{Emmagatzemament:}
    \begin{itemize}
        \item 16 GB de disc de sistema
        \item 256 GB de disc SSD per dades
    \end{itemize}
\end{itemize}

\noindent
Aquest disposa de dos usuaris, \textit{tfe}, per l’ús diari, i un altre, amb privilegis d’administrador, \textit{root}, en cas que es necessitin aquests.
L’accés es realitza a través de l’usuari tfe, utilitzant una connexió \gls{ssh}:

\begin{verbatim}
    $ ssh tfe@ostia.epsevg.upc.edu
\end{verbatim}

\noindent
Les tasques principals realitzades són:

\begin{itemize}
    \item Desenvolupament del codi d’\gls{OSTIA}, utilitzant el sistema de control de versions git.
    \item Anàlisis dels logs d’\gls{UPCommons}, amb registres des del 2006 fins al 2023.
    \item Processament i abocament dels logs a la base de dades InfluxDB, ubicada al mateix servidor.
    \item Descàrrega de totes les metadades del servidor \gls{OAI} sbivdev1.
    \item Anàlisis, processament i abocament de les metadades a una base de dades MongoDB.
    \item Anàlisi quantitativa i qualitatiu del conjunt de les dades, amb el suport l’eina Grafana.
\end{itemize}

\clearpage

\section*{Configuració}\label{sec:server-configuration}

Els principals serveis utilitzats durant la realització del projecte han sigut:

\begin{itemize}
    \item \textbf{InfluxDB}
    \begin{itemize}
        \item Base de dades on hem emmagatzemat els logs.
        \item {
            Incorporem també el \textit{\gls{plugin}} d’InfluxDB anomenat Telegraf.
            Encara que el seu ús no és imperatiu pels nostres objectius, pot ser útil per recol·lectar dades del sistema.
        }
    \end{itemize}
    \item \textbf{MongoDB}
    \begin{itemize}
        \item Base de dades utilitzada per l’emmagatzematge de les metadades.
        \item El mateix servei no ofereix cap interfície gràfica, així doncs afegim el suport de mongodb-expres.
    \end{itemize}
    \item \textbf{Grafana}
    \begin{itemize}
        \item Principal eina d’observabilitat, emprada tant per l’anàlisi com per la visualització gràfica.
    \end{itemize}
\end{itemize}

\noindent
Per cadascun d’aquests serveis definirem un fitxer de configuració, que configurarà cada contenidor, que en aquest cas, són de tipus \textit{\gls{Docker}}. \\

\noindent
Com disposem de diverses aplicacions, la manera més senzilla és utilitzar el sistema per configurar i executar aplicacions multi-contenidor per antonomàsia, que és \textit{\gls{docker-compose}}. \\

\noindent
Aquest fitxer acostuma a tenir aquesta estructura:

\begin{itemize}
    \item Definició del nom del servei.
    \item Imatge de \textit{\gls{Docker}} que es vol utilitzar.
    \item Els ports, definint el port de la màquina host i el del contenidor.
    \item El fitxer de configuració del servei.
    \item La xarxa interna del servei de \textit{\gls{Docker}} que adoptarà el servei.
    \item Altres opcions.
\end{itemize}

\clearpage

\noindent
A l'exemple següent es troba la configuració específica per la base de dades InfluxDB.

\begin{figure}[htbp]
    \centerline{\includegraphics[width=\textwidth]{figures/docker-compose-influxdb}}
    \captionsetup{justification=centering}
    \caption{Configuració del servei d'InfluxDB.}\label{fig:docker-compose-influxdb}
    \source{Repositori de codi del projecte.}
\end{figure}

\begin{tcolorbox}[colback=blue!5!white, colframe=blue!75!black, title=Fitxers de configuració]
    Habitualment aquests fitxers solen incloure credencials privades d'aquests serveis.
    Com a bona pràctica, afegeix el sufix \textit{.env} al fitxer i no el pugis a Github, omitin-lo mitjançant el fitxer \textit{.\gls{gitignore}}.
\end{tcolorbox}
\vspace{1em}

\noindent
Un cop definits tots els serveis, es procedeix amb la instal·lació.
La comanda següent, descarrega les imatges de \textit{\gls{Docker}} en cas que aquestes no estiguin presents al sistema,
crea les xarxes virtuals pels contenidors i finalment, configura els serveis.

\begin{verbatim}
    $ docker compose up -d
\end{verbatim}

\begin{tcolorbox}[colback=red!5!white, colframe=red!75!black, title=Atenció]
    Aquesta comanda s'ha d'executar en el mateix directori on es troba el fitxer \textit{\gls{docker-compose}}.
\end{tcolorbox}

\clearpage

\section*{Accés als serveis}\label{sec:server-access}

\noindent
Per accedir als diferents serveis que es troben al servidor, utilitzarem un túnel ssh~\cite{tunel-ssh}.
Aquest consisteix en xifrar tota comunicació entre client i servidor. \\

\noindent
Per exemple, el contingut de la interfície gràfica de Grafana, accessible a través del port tres mil del servidor,
serà accessible pel port tres mil de la màquina \textit{host} amb els següents passos:

\begin{verbatim}
    $ ssh -N -L localhost:3000:ostia.epsevg.upc.edu:3000 \
        tfe@ostia.epsevg.upc.edu
\end{verbatim}
\begin{verbatim}
    $ open https://localhost:3000
\end{verbatim}