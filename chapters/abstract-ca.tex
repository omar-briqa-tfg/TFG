\chapter*{Resum}\label{ch:abstract-ca}

En el context actual de la ciència de dades, cada registre d'un esdeveniment és crucial.
Profunditzar sobre aquesta informació pot proporcionar informació molt valuosa.
En el nostre cas, se'ns ha donat accés als \textit{logs} del servidor UPCommons, el portal del coneixement obert de la Universitat Politècnica de Catalunya.

\noindent \\
Els \textit{logs} són el registre d'accés, la petjada digital de cada usuari de la plataforma.
Per cada examen, documentació, treball, vídeo o altre recurs consultat, aquest accés queda enregistrat.
El nostre objectiu és recopilar totes aquestes dades i convertir-les en informació significativa.

\noindent \\
Aquest procés consta de tres passos.
En primer lloc, entendre la semàntica i la sintaxi dels registres.
Quin tipus d'informació tractarem, on es troba, quina informació inclou, i com ho analitzarem.
Tot això tenint en compte l'abast de la tasca, tots els registres d'accés des de l'any 2006 fins al 2023, en concret, 1.922.392.760 registres d'entrada.

\noindent \\
En segon lloc, un cop aclarida la semàntica dels \textit{logs}, es necessita una solució d'emmagatzematge per dur a terme una anàlisi sobre la informació prèviament estudiada.
Hem de filtrar i decidir què emmagatzemarem, en quin format s'emmagatzema, i el més important, on ho emmagatzemarem.

\noindent \\
Finalment, hem pogut construir una eina de codi obert que pot analitzar i emmagatzemar tots els registres d'accés.
Podem definir un cas d'ús per analitzar una determinada característica.
Utilitzant l'eina d'observabilitat Grafana, els resultats es poden representar visualment.

\noindent \\
Per exemple, podem comprovar quin recurs és el més consultat dins d'un període de temps i representar la seva evolució.
Si peticions malicioses han infectat el servidor, podem utilitzar aquesta eina per analitzar els símptomes.
També relacionar el recurs amb les seves metadades: autors, llicència, llengua, etcètera.

\noindent \\
El valor que proposem és una eina que es pot utilitzar per a diferents finalitats, i un punt de partida per a futures investigacions.

\clearpage
\section*{Paraules clau}\label{sec:keywords-ca}
\begin{multicols}{3}
    \begin{itemize}
        \item DSpace
        \item Metadades
        \item UPCommons
    \end{itemize}
    \columnbreak
    \begin{itemize}
        \item Registre d’accés
        \item Anàlisi de dades
        \item Python
    \end{itemize}
    \columnbreak
    \begin{itemize}
        \item Grafana
        \item InfluxDB
        \item MongoDB
    \end{itemize}
\end{multicols}