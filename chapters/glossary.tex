\newglossaryentry{apt}{
    name            = apt,
    description     = {Advanced Package Tool}
}

\newglossaryentry{log}{
    name            = log,
    description     = {Registre d'accés a un servidor, aplicatiu, servei, etc}
}

\newglossaryentry{ssh}{
    name            = ssh,
    description     = {Programari utilitzat per enviar comandes a una màquina remotament a través d'una xarxa insegura}
}

\newglossaryentry{OSTIA}{
    name            = OSTIA,
    description     = {Open Science Toolkit Information Access. Denominació de l'eina dissenyada per analitzar i processar les dades d'acces al repositori d'UPCommons. També és el nom del servidor que hostatja tots els recursos de l'eina}
}

\newglossaryentry{UPCommons}{
    name            = UPCommons,
    description     = {Repositori institucional d'accés obert de la Universitat Politècnica de Catalunya}
}

\newglossaryentry{OAI}{
    name            = OAI,
    description     = {Open Archives Initiative, normalment associat amb OAI-PMH}
}

\newglossaryentry{OAI-PMH}{
    name            = OAI-PMH,
    description     = {Open Archives Initiative Protocol for Metadata Harvesting. Estàndard d’interoperabilitat desenvolupat per l’intercanvi i la difusió de metadades}
}

\newglossaryentry{plugin}{
    name            = plugin,
    description     = {Complement, extensió}
}

\newglossaryentry{Docker}{
    name            = Docker,
    description     = {Plataforma per la construcció, execució i desenvolupament d'aplicacions basades en contenidors}
}

\newglossaryentry{docker-compose}{
    name            = docker-compose,
    description     = {Eina per la definició i execució d'aplicacions multicontenidor}
}

\newglossaryentry{gitignore}{
    name            = gitignore,
    description     = {Al sistema de control de versions \textit{git}, especifica els fitxers intencionalment ignorats}
}

\newglossaryentry{GitHub}{
    name            = GitHub,
    description     = {Plataforma de desenvolupament col·laboratiu de codi font, emmagatzematge de projectes de programari, control de versions i gestió de codi font}
}

\newglossaryentry{DSpace}{
    name            = DSpace,
    description     = {Plataforma de codi obert utilitzada per construir i gestionar repositoris digitals}
}

\newglossaryentry{DIM}{
    name            = DIM,
    description     = {DSpace Intermediate Metadata}
}

\newglossaryentry{TCP}{
    name            = TCP,
    description     = {Transmission Control Protocol}
}

\newglossaryentry{HTTP}{
    name            = HTTP,
    description     = {Hypertext Transfer Protocol}
}

\newglossaryentry{IP}{
    name            = IP,
    description     = {Internet Protocol}
}

\newglossaryentry{ISO}{
    name            = ISO,
    description     = {Internacional Organization for Standardization}
}

\newglossaryentry{POSIX}{
    name            = POSIX,
    description     = {Portable Operating System Interface X}
}

\newglossaryentry{NAT}{
    name            = NAT,
    description     = {Network Address Translation}
}

\newglossaryentry{handle}{
    name            = handle,
    description     = {En el context de DSpace, és un identificador persistent utilitzat per referenciar de manera única objectes en repositoris}
}

\newglossaryentry{CSS}{
    name            = CSS,
    description     = {Cascading Style Sheets}
}

\newglossaryentry{bitstream}{
    name            = bitstream,
    description     = {En el context de DSpace, model que representa un arxiu qualsevol mitjançant un flux de bits}
}

\newglossaryentry{UUID}{
    name            = UUID,
    description     = {Universally Unique Identifier}
}

\newglossaryentry{JSON}{
    name            = JSON,
    description     = {JavaScript Object Notation}
}

\newglossaryentry{API}{
    name            = API,
    description     = {Application Programming Interface}
}

\newglossaryentry{timestamp}{
    name            = timestamp,
    description     = {Registre de la data i l'hora en què ocorre un esdeveniment}
}

\newglossaryentry{gzip}{
    name            = gzip,
    description     = {Programari utilitzat per la compressió d'arxius. També fa referència al format de dades comprimides amb aquest programari}
}

\newglossaryentry{XML}{
    name            = XML,
    description     = {Extensible Markup Language}
}

\newglossaryentry{git}{
    name            = git,
    description     = {Sistema de control de versions}
}

\newglossaryentry{commit}{
    name            = commit,
    description     = {Registre dels canvis realitzats en una base de codi en sistemes de control de versions com \textit{git}}
}

\newglossaryentry{DNS}{
    name            = DNS,
    description     = {Domain Name System}
}

\newglossaryentry{URL}{
    name            = URL,
    description     = {Uniform Resource Locator, tipus específic d'URI}
}

\newglossaryentry{Dublin Core}{
    name            = Dublin Core,
    description     = {També conegut com a Dublin Core Metadata Terms, és un vocabulari format per un conjunt de 15 elements estàndard per descriure recursos digitals, facilitant la catalogació i recuperació d'informació}
}

\printglossary[title={Glossari}]