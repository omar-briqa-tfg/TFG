\chapter*{Resumen}\label{ch:abstract-es}

En el contexto actual de la ciencia de datos, cada registro de un acontecimiento es crucial.
Profundizar sobre esta información puede proporcionar información muy valiosa.
En nuestro caso, se nos ha dado acceso a los \textit{logs} del servidor UPCommons, el portal de acceso abierto al conocimiento de la Universitat Politècnica de Catalunya.

\noindent \\
Los \textit{logs} son el registro de acceso, la huella digital de cada usuario de la plataforma.
Por cada examen, documentación, trabajo, vídeo u otro recurso consultado, este acceso queda registrado.
Nuestro objetivo es recopilar todos estos datos y convertirlos en información significativa.

\noindent \\
Este proceso consta de tres pasos.
En primer lugar, entender la semántica y la sintaxis de los registros.
Qué tipo de información trataremos, donde se encuentra, qué información incluye, y como lo analizaremos.
Todo esto teniendo en cuenta el tamaño de la tarea, todos los registros de acceso desde el año 2006 hasta el 2023, en concreto, 1.922.392.760 registros de entrada.

\noindent \\
En segundo lugar, una vez aclarada la semántica de los \textit{logs}, se necesita una solución de almacenamiento para llevar a cabo un análisis sobre la información previamente estudiada.
Tenemos que filtrar y decidir qué almacenaremos, en qué formato se almacena, y lo más importante, donde lo almacenaremos.

\noindent \\
Finalmente, hemos podido construir una herramienta de código abierto que puede analizar y almacenar todos los registros de acceso.
Podemos definir un caso de uso para analizar una determinada característica.
Utilizando la herramienta de observabilidad Grafana, los resultados se pueden representar visualmente.

\noindent \\
Por ejemplo, podemos comprobar qué recurso es lo más consultado dentro de un periodo de tiempo y representar su evolución.
Si peticiones maliciosas han infectado el servidor, podemos utilizar esta herramienta para analizar los síntomas.
También relacionar el recurso con sus metadatos, autor/s, licencia, lengua, etcétera.

\noindent \\
El valor que proponemos es una herramienta que se puede utilizar para diferentes objetivos, y un punto de partida para futuras investigaciones.

\clearpage
\section*{Palabras clave}\label{sec:keywords-es}
\begin{multicols}{3}
    \begin{itemize}
        \item DSpace
        \item Metadatos
        \item UPCommons
    \end{itemize}
    \columnbreak
    \begin{itemize}
        \item Registro de acceso
        \item Análisis de datos
        \item Python
    \end{itemize}
    \columnbreak
    \begin{itemize}
        \item Grafana
        \item InfluxDB
        \item MongoDB
    \end{itemize}
\end{multicols}