\chapter*{Conclusions}\label{ch:conclusion}
\addcontentsline{toc}{chapter}{Conclusions}

Començant per l'anàlisi dels registres d'accés, hem posat el focus en la modularitat del disseny, i no acoblar aquest amb la implementació.
Principalment, és perquè les decisions del que com s'ha analitzat poden variar si el cas d'ús és totalment diferent.

\noindent \\
Per exemple, si necessitéssim afegir un paràmetre addicional als \textit{\gls{log}s} per a una anàlisi específica o volem desglossar els camps del User Agent, aquestes decisions no tindran conseqüències negatives ni efectes ``dominó'' en altres parts del sistema.

\noindent \\
Per això mateix no hem explicat al detall com s'ha fet, per exemple, per extraure l'adreça IP dels \textit{\gls{log}s}, o com extraiem la versió d'\gls{HTTP} emprada.
Les decisions d'implementació són molt variades.
En canvi, definir els paràmetres i la semàntica d'un \textit{\gls{log}} proporciona informació més rellevant.

\noindent \\
Definint els passos a partir de l'anàlisi dels \textit{\gls{log}s}, hem tractat i emmagatzemat tots els registres excepte aquells que són d'accessos a recursos web.
Això ha resultat en poder descartar i no emmagatzemar un 22\% del \textit{\gls{log}s} (430.789.238 registres).

\noindent \\
L'evolució del nombre de \textit{\gls{log}s} al llarg dels anys es manté creixent de forma estable.
Cal destacar, però, que de l'any 2012 amb 80 milions d'accessos a \gls{UPCommons}, el 2013 hi va haver 33 milions.
Aquesta alteració va continuar l'any següent amb 28 milions d'accessos, però tornà a la tendència habitual l'any següent.

\noindent \\
Abocar els registres a una base de dades de sèries temporals ens permet explotar totes les seves funcionalitats per dur a terme anàlisis temporals detallats.
Hem planificat aquest procés per enviar les dades en tandes a InfluxDB.

\noindent \\
Com a resultat, hem processat 1.922.392.760 entrades de \textit{\gls{log}s} en un temps d'aproximadament 33 hores, mitjançant un procés completament automatitzat.
Un possible punt de treball futur és poder paral·lelitzar aquest procés, pel fet que els \textit{\gls{log}s} són independents entre si.
Sistemes com Hadoop o Spark estan especialitzats en aquests processos.

\noindent \\
Utilitzem el protocol OAI-PMH, i la llibreria Sickle.
La complexitat d'aquest protocol fa possible utilitzar peticions d'\gls{HTTP} directes contra el servidor OAI, ometent l'ús de llibreries externes.
Com les metadades són obtingudes en format XML, el pas de la conversió a un document \gls{JSON} és directe.
Aquest ha sigut el motiu principal per utilitzar una base de dades basada en documents.

\noindent \\
Els camps de les metadades poden ser de múltiple valor, l’autor, les paraules clau, la matèria relacionada amb el recurs, un altre argument a favor de l’ús de bases de dades basada en documents, gràcies a la seva flexibilitat en l’esquema de dades.
Comparat amb el número de \textit{\gls{log}s}, l'abocament és molt més ràpid.
Hem afegit 242.555 entrades en 38 segons a la base de dades MongoDB.

\noindent \\
Tot aquest processament s'ha dut a terme en el servidor específic pel desenvolupament del projecte.
Allà disposem de tots els \textit{\gls{log}s} emmagatzemats en sistema de fitxers, a més de tenir-los a la base de dades.
El mateix passa amb les metadades.
Això significa que els processos d'abocament s'han fet de manera “interna”.
El fet de disposar la base de dades en una altra ubicació, ja sigui en un altre servidor intern fora de la xarxa local, o en un servei al núvol, té certes implicacions.

\noindent \\
S'ha de tenir controlada la capacitat de xarxa, adaptar els clients de cada servei al límit de peticions per segon, el nombre màxim d'unitats d'escriptura, i altres opcions.
Una configuració inapropiada podria resultar en errors de processament, i si estem parlant d'automatitzar la tasca, el temps d'abocament podria augmentar dràsticament.
En el pitjor cas, si estem utilitzant algun servei que necessiti subscripció econòmica, la factura final ens augmentarà.

\noindent \\
Com ja hem vist als diferents casos d'ús, els paràmetres escollits per processar els \textit{\gls{log}s} i emmagatzemar-los ens han permès dur algun tipus d'anàlisis de forma immediata.

\noindent \\
Aquests són aquells que utilitzen els camps dels \textit{\gls{log}s}, ja que les cerques estan optimitzades gràcies a la indexació.
El desenvolupament d'aquells casos d'ús relacionats amb el codi d'estat, si el contingut del \textit{\gls{log}} és correcte o diferent de l'habitual, el tipus de petició d'\gls{HTTP} es processen veloçment.
L'exemple el tenim amb el segon cas d'ús on analitzem els registres d'accés que tenen el seu contingut alterat de tot l'any 2023.

\noindent \\
Després tenim aquells casos d'ús on hem d'analitzar els camps dels \textit{\gls{log}s}.
Aquests valors són molt diversos, i no els podem indexar.
Són per exemple, el valor complet del log, o l'identificador del recurs que s'ha accedit.

\noindent \\
Com a conseqüència, aquesta anàlisi es basarà majoritàriament en un recorregut complet de totes les dades per obtenir-ne resultats.
El temps de processament és directament proporcional a la quantitat total de \textit{\gls{log}s} processats.
L'exemple el tenim amb el segon cas d'ús on analitzem els registres d'accés que tenen el seu contingut mal format de tot l'any 2023.

\noindent \\
No és casualitat que només analitzem al primer cas d'ús, que es basa a obtenir el recurs més accedit, el primer mes del 2023.
Aquest procés triga 30 vegades més que el segon cas d'ús, encara que estiguem tractant d'11 vegades menys dades (un vs\. dotze mesos d'accessos).

\noindent \\
El tercer tipus de cas d'ús consisteix a relacionar els recursos amb les seves metadades.
En aquesta etapa analitzem específicament els registres d'accés a recursos d'\gls{UPCommons} per relacionar-los amb les seves metadades.
Aquests casos es basen a cercar a MongoDB el valor d'aquestes metadades.

\noindent \\
L'exemple mostra un cas en què hem filtrat els recursos pel centre docent on s'han adscrit, per després obtenir-ne la tipologia d'accés del recurs.
Com era d'esperar, la majoria d'accessos eren a recursos d'accés obert o restringit a la comunitat acadèmica, ja que aquests són els que poden accedir la majoria d'usuaris d'\gls{UPCommons}.

\noindent \\
Aquesta recopilació de dades s'ha fet consultant tots els accessos a la base de dades dels \textit{\gls{log}s}, i per cada un d'ells consultant la base de dades de les metadades.
Caldria valorar l'opció de centralitzar aquesta informació en una mateixa font de dades.
Una tasca gens trivial, ja que tractem de dos tipus diferents de dades, amb propòsits diferents.

\noindent \\
La solució presa és realitzar l'anàlisi programàticament i afegir els resultats a InfluxDB, en cas de ser una anàlisi temporal.
I després consumir aquesta informació amb Grafana.

\noindent \\
Grafana ens ha permès analitzar visualment aquests casos, obtenint informació de les nostres fonts de dades.
Aquesta eina és la més recomanada degut al seu ús popular, tant en la versió gratuïta com en la seva versió “Cloud”.
La gran majoria de bases de dades que utilitzaríem per emmagatzemar els \textit{\gls{log}s} i les metadades tenen suport per a Grafana.

\noindent \\
Arribats a aquest punt, hem vist com a partir d'analitzar els registres d'accés a \gls{UPCommons}, entendre el funcionament de les metadades dels recursos, afegir aquesta informació a un gestor de bases de dades, hem pogut desenvolupar uns casos d'ús base que demostren que aquesta eina pot donar suport per analitzar les diferents casuístiques proposades per l'usuari de l'eina.

