\chapter*{Introducció}\label{ch:introduction}
\addcontentsline{toc}{chapter}{Introducció}

L'objectiu d'aquest projecte és desenvolupar una eina de codi obert que, a partir de l'anàlisi dels registres d'accés de la plataforma \gls{UPCommons}, converteixi aquesta informació específica del \textit{software} que utilitza \gls{UPCommons}, a un conjunt de dades més comprensibles.
En altres paraules, volem donar sentit i estructura als \textit{\gls{log}s} per extreure informació valuosa.

\noindent \\
La nostra recerca inclou els accessos de divuit anys, que abasten des de l'any 2006 fins al 2023, amb registres procedents des de qualsevol ubicació, donat que es tracta d'una plataforma en línia.
Aquests registres consisteixen principalment en consultes de recursos d'\gls{UPCommons}, malgrat que hi ha més tipus d'accessos: cerques, comandes del servidor internes, redireccions, i molt més.

\noindent \\
Analitzar aquest conjunt de dades representa un gran repte.
El funcionament intern d'\gls{UPCommons} i els seus recursos, la gran quantitat d'entrades a processar i els mètodes de processament van ser incerteses inicials a les que respondrem al llarg d'aquest projecte.
La tasca de desenvolupar una eina que després pot ser de gran ajuda per a moltes persones, pel fet que \gls{DSpace} és un aplicatiu molt utilitzat~\cite{eprints:roar}, és un factor de motivació.

\noindent \\
Hem fragmentat el nostre projecte en tres etapes ben diferenciades.
La primera se centra en l'anàlisi dels registres d'accés.
Definirem clarament què esperem d'un \textit{\gls{log}}, quin contingut ha de contenir i quina informació ha de proporcionar.

\noindent \\
Seguidament, emmagatzemarem aquesta informació a una base de dades de sèries temporals, InfluxDB, ja que els \textit{\gls{log}s} són registres d'esdeveniments que ocorren en marques de temps específiques.
A més a més, descarregarem totes les metadades dels recursos d'\gls{UPCommons} d'un servidor OAI (Open Archives Initiative) a una base de dades basada en documents, MongoDB.

\noindent \\
Finalment, amb les dades processades i emmagatzemades en un gestor de bases de dades, procedirem a definir uns casos d'ús que serviran de base i demostraran la proposta de valor de la nostra eina.
Com a eina de suport, utilitzarem el programari d'observabilitat Grafana.

\noindent \\
Per a cada etapa, hem considerat diferents programaris, avaluant els punts a favor i en contra de cadascun.
Aquest procés ens ha permès seleccionar les eines més adequades per a les necessitats específiques de cada fase del projecte.

\noindent \\
Aquesta descomposició conceptual també ens ha facilitat modular les tasques al llarg del temps.
El primer mes de treball va servir com a presa de contacte, així com una anàlisi general i definició de l'abast del projecte.
Després, els mesos d'abril, maig i juny van ser enfocats en l'anàlisi i el processament de les dades, l'emmagatzemament de les dades i la definició dels casos d'ús, respectivament.

\noindent \\
Per documentar-nos, hem utilitzat documents interns proporcionats pels responsables d'\gls{UPCommons}.
A més, hem revisat detalladament la documentació de cada servei i programari utilitzats per analitzar les seves característiques.

\noindent \\
Elaborar informes periòdics per documentar el progrés, les accions realitzades, els obstacles i els punts d'actuació ha estat el procediment principal de treball.
Respecte al codi, s'ha desenvolupat seguint les pràctiques recomanades de l'enginyeria de programari i del cicle de vida del desenvolupament del programari.
El repositori es troba públicament a \textit{\gls{GitHub}} sota una llicència MIT.

\noindent \\
El primer capítol de la memòria descriu la fase de processament dels \textit{\gls{log}s}: l'anàlisi preliminar, el disseny i la implementació del processament.
El següent capítol aborda l'emmagatzemament dels \textit{\gls{log}s} i les metadades un cop aquesta informació ha sigut processada.
Al darrer capítol utilitzem aquesta informació a diversos casos d'ús descrivint diverses característiques de les dades.

\noindent \\
Per últim, trobareu diversos annexes.
El primer tracta sobre la metodologia de desenvolupament i l'estructura del codi de l'eina desenvolupada: Open Science Toolkit Information Acces.
Seguidament tenim un annex que conté una descripció detallada del servidor de treball que allotja tots els requeriments de l'eina, bases de dades, repositori de codi i programari.

\noindent \\
El tercer annex és un apartat específic on es detalla d'una manera instructiva com configurar el programari d'observabilitat Grafana per consumir de la nostra base dades InfluxDB.
El penúltim annex consta de la fragmentació del resultat de l'abocament dels \textit{\gls{log}s} a la base de dades, any per any.

\noindent \\
Finalment, el darrer annex tracta del procés d'anonimització dels \textit{\gls{log}s}, on s'inclou el document del pla de gestió de dades.