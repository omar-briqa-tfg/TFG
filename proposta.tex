\documentclass[11pt]{report}

\usepackage{amsmath}
\usepackage{caption}
\begin{document}

\begin{center}
    \LARGE
    \textbf{Proposta del TFE}

    \vspace{1cm}
    
    \Large
    \textbf{Disseny d'una eina per al tractament i visualització de les dades d'accés a un repositori} \\
\end{center}

\noindent \\
L'objectiu d'aquest projecte és, a partir de l'anàlisi dels \textit{logs} d'un repositori, desenvolupar una eina que permeti visualitzar quins continguts són els més consultats, quines són les seves llicències, quins temes són els més cercats i qui (cercadors, Bots, organitzacions, persones\dots) fa les consultes, entre altres aspectes, i tot això tenint present una perspectiva temporal.

\noindent \\
Per aconseguir aquest objectiu general es defineixen aquests objectius:

\begin{itemize}
    \item Desenvolupar una eina per automatitzar el processament dels \textit{logs}.
    \item Definir una o més bases de dades on guardar els \textit{logs} un cop processats.
    \item Seleccionar el tipus més adequat de base de dades donat el volum d'aquestes i les consultes que es volen fer.
    \item Creació de la(es) bases de dades.
    \item Adequació, test i validació de l'eina d'automatització del processament del \textit{logs} contra la(es) base(s) de dades.
    \item Seleccionar una eina o eines de visualització de les dades.
    \item Instal·lació, configuració i test de l'eina de visualització.
    \item Generació d'informes (visualització) a partir de la informació recollida a la(es) base(s) de dades.
\end{itemize}

\end{document}