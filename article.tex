\documentclass[lettersize,journal]{IEEEtran}
\usepackage{amsmath,amsfonts}
\usepackage{array}
\usepackage{textcomp}
\usepackage{url}
\usepackage{verbatim}
\usepackage{graphicx}

\renewcommand{\abstractname}{Resum}
\renewcommand{\IEEEkeywordsname}{Paraules clau}

\begin{document}

\title{Disseny d'una Eina per al Tractament i \\ Visualització de les Dades d'Accés a un Repositori}
\author{\textbf{Omar Briqa} \\ \vspace{5pt} Escola Politècnica Superior d'Enginyeria de Vilanova i la Geltrú}
\maketitle

\begin{abstract}
   Resum
\end{abstract}

\begin{IEEEkeywords}
    .
\end{IEEEkeywords}

\section{Introduction}

\IEEEPARstart{T}{his} file is intended to serve as a ``sample article file''
for IEEE journal papers produced under \LaTeX\ using
IEEEtran.cls version 1.8b and later. The most common elements are covered in the simplified and updated instructions in ``New\_IEEEtran\_how-to.pdf''. For less common elements you can refer back to the original ``IEEEtran\_HOWTO.pdf''. It is assumed that the reader has a basic working knowledge of \LaTeX. Those who are new to \LaTeX \ are encouraged to read Tobias Oetiker's ``The Not So Short Introduction to \LaTeX ,'' available at: \url{http://tug.ctan.org/info/lshort/english/lshort.pdf} which provides an overview of working with \LaTeX.

\begin{thebibliography}{1}
\bibliographystyle{IEEEtran}

\bibitem{ref1}
    {\textit{Mathematics Into Type}}. American Mathematical Society. [Online]. Available: https://www.ams.org/arc/styleguide/mit-2.pdf
\end{thebibliography}

\end{document}
