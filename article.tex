\documentclass[lettersize,journal]{IEEEtran}
\usepackage{amsmath,amsfonts}
\usepackage{array}
\usepackage{textcomp}
\usepackage{url}
\usepackage{verbatim}
\usepackage{graphicx}
\usepackage[catalan]{babel}
\usepackage{lipsum}

\renewcommand{\abstractname}{Resum}
\renewcommand{\IEEEkeywordsname}{Paraules clau}

\begin{document}

\title{Disseny d'una Eina per al Tractament i \\ Visualització de les Dades d'Accés a un Repositori}
\author{\textbf{Omar Briqa} \\ \vspace{5pt} Escola Politècnica Superior d'Enginyeria de Vilanova i la Geltrú}
\maketitle

\begin{abstract}
    .
\end{abstract}

\begin{IEEEkeywords}
    DSpace, Metadades, UPCommons, Grafana, InfluxDB, MongoDB, Registre d’accés.
\end{IEEEkeywords}


\section{Introducció}\label{sec:introduction}
\IEEEPARstart{I} ntroducció
\lipsum[1]

\section{Processament de les dades}\label{sec:data-processing}
La primera etapa del nostre projecte consisteix a processar les dades.
Volem respondre a preguntes com: què són aquestes dades, com estan estructurades, quina informació proporcionen, com les podríem tractar\dots
Durant aquest procés, analitzarem tots els registres d'accés, emmagatzemats al nostre servidor de treball.
Concretament, parlem dels registres d'accés des del primer dia de l'any 2006 fins al darrer accés de l'any 2023.


A causa de la gran quantitat de dades, hem de dissenyar un procés que analitzi tots aquests registres de manera automatitzada.
Aquest procés ha de tractar diverses casuístiques dels registres, ja que els servidors no només registren accessos a la plataforma d'UPCommons, sinó que moltes més coses.
Accessos a recursos web, repeticions de registres, alguns amb el seu format alterat, són alguns casos que hem anat trobant pel camí.
Un cop aclarits tots els casos, proposem un disseny modular, independent de la seva implementació, per dur a terme aquesta anàlisi.


L'objectiu d'aquest és convertir els \textit{logs} del format específic amb el qual s'enregistren a la plataforma d'UPCommons, a un altre més operable, on també hi hem afegit informació significativa que ens serà útil per als passos posteriors.
Saber si el \textit{log} és una cerca a la plataforma, un accés a un recurs o d'altre tipus, i marcar-ho al mateix registre pot simplificar molt el seu tractament.


Per altra banda, processarem també les metadades.
Aquestes són les etiquetes presents a cadascun dels recursos d'UPCommons, amb l'objectiu d'enriquir la informació d'aquests.
Els autors, la descripció del recurs, la data de pujada, el departament responsable, el tipus de document o més, es troben inclosos.
La plataforma d'UPCommons està basada en el programari de codi obert DSpace, plataforma per la construcció i gestió de repositori digitals, una de les més utilitzades [REF].
El format de les metadades és el DIM (DSpace Intermediate Metadata), que es basa en l'agregació d'un qualificador al format Dublin Core ja existent.
Per exemple, l'autor principal d'un recurs ve definit per la metadada:
\begin{center}
    {dc.contributor.author}
\end{center}


Les metadades les podrem trobar a un servidor de desenvolupament on només hi són emmagatzemadaes aquestes.
Utilitzarem el protocol OAI-PMH (Open Archives Initiative Protocol for Metadata Harvesting) per descarregar aquestes.
Una característica interessant d'aquest servidor, i que no tots els servidors OAI tenen, és la funcionalitat del control de flux mitjançant \textit{resumptionToken}.
Els verbs del protocol OAI-PMH per obtenir les metadades no poden obtenir totes les metadades d'una tirada, ja que no és gens pràctic a causa del seu gran nombre.
Com a conseqüència, retornen un valor que indica en quin punt de la descàrrega ets, permetent la modularització de l'obtenció de les metadades.


Implementarem aquest procés amb la llibreria Sickle [REF], encara que gràcies a la simplicitat d'aquest protocol podríem dur a terme sense la necessitat de llibreries externes.
Com a primera fase, les estructurarem localment en el nostre sistema de fitxers.
Més endavant les migrarem a un sistema gestor de bases de dades.


\section{Emmagatzemament de les dades}\label{sec:data-storing}
\lipsum[1]

\section{Servidor de treball}\label{sec:server-ostia}
\lipsum[1]

\section{Anàlisi i visualització de les dades}\label{sec:data-visualization}
\lipsum[1]

\section{Conclusions}\label{sec:conclusions}
\lipsum[1]

\begin{thebibliography}{1}
\bibliographystyle{IEEEtran}

\bibitem{ref1}
    {\textit{Mathematics Into Type}}. American Mathematical Society. [Online]. Available: https://www.ams.org/arc/styleguide/mit-2.pdf
\end{thebibliography}

\end{document}
