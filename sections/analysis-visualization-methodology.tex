\section{Metodologia}\label{sec:analysis-visualization-methodology}

\noindent
Com veurem més endavant, els casos se solen definir de les següents formes:

\begin{itemize}
    \item S'ha complert aquest criteri en algun moment.
    \item Per quin període temporal es compleix el criteri X, Y i Z.
    \item Donat aquest conjunt de dades vull extraure aquesta informació per aquest període de temps
\end{itemize}

\noindent
On els criteris i els períodes de temps poden ser qualssevol.

\noindent \\
Extraurem la informació de les nostres bases de dades: InfuxDB on es troben emmagatzemats els \textit{\gls{log}s}, i MongoDB on hi són emmagatzemades les metadades dels recursos d'\gls{UPCommons}.

\noindent \\
Amb l'ajuda de \textit{Grafana} representarem de forma visual el nostre cas d'ús i les conclusions extretes d'aquest.

\noindent \\
Depenent del volum del conjunt de dades,  farem l'anàlisi directament sobre \textit{Grafana} si el sistema pot processar-les en temps real.
En cas contrari, l'anàlisi serà aïllat per després mostrar aquests resultats visualment.
