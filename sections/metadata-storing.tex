\section{Metadades}\label{sec:metadata-storing}

Aprofitant el que conjunt total de metadades té un volum tractable (648 MB), un pas anterior del bolcat a una base de dades ha sigut emmagatzemar-les al sistema. \\

\noindent
D’aquesta manera evitem tornar a fer les peticions al servidor en cas que el bolcat falli, o les dades pateixin algun tipus d’alteració o corrupció. \\

\noindent
L’estructura consisteix en un directori per cada miler de metadades, i deu fitxers amb cen cadascun, incloses en aquest.

\begin{verbatim}
  - batch_0_1000
    |-- 0_100.metadata.json
    |-- 100_200.metadata.json
    |-- ...
  - batch_1000_2000
    |-- 1000_1100.metadata.json
    |-- ...
\end{verbatim}

\noindent \\
\subsection{Destí: MongoDB}\label{subsec:metadata-db-mongodb}

L'elecció de la base de dades on emmagatzemarem les metadades depén de l'estructura d'aquestes.
Com el servidor \gls{OAI} retorna les dades en format \gls{XML}, i la conversió a un document \gls{JSON} és quasi trivial, utilitzarem \textit{MongoDB}.

\noindent \\
MongoDB és una base de dades que ofereix un model de dades flexible, escalabilitat horitzontal i suport per a transaccions.
Les seves característiques principals són:

\begin{itemize}
  \item Base de dades no relacional de codi obert, orientada a documents.
  \item Com alguns camps no són únics com per exemple, el dc.contributor, necessitem un gestor que ens ofereixi flexibilitat al seu esquema de dades.
  \item Es pot utilitzar com a font de dades i incorporar-la a \textit{Grafana}, a través d’un \gls{plugin}.
\end{itemize}

\noindent
Com prèviament hem vist, les metadades s’estructuren de la següent manera: \textit{schema.element.qualifier}.
A causa que MongoDB no suporta punts (.) a les seves claus~\cite{mongodb:key-restrictions}, s’ha hagut de fer una petita transformació:

\begin{center}
  schema\textbf{.}element\textbf{.}qualifier → schema\textbf{-}element\textbf{-}qualifier
\end{center}

\clearpage

\subsection{Estadístiques}\label{subsec:metadata-statistics}

L'abocament de les metadades s'ha dut a terme en dues fases:

\begin{enumerate}
  \item Descàrrega i l'emmagatzematge al sistema, concretament a \\ \texttt{/mnt/working/metadata}.
  \item Consumir d'aquest directori ha sigut la font per injectar les dades a \\ MongoDB.
\end{enumerate}

\noindent \\
El resultat s'il·lustra mitjançant les següents dues taules: \\

\begin{table}[h!]
    \centering
    \begin{adjustbox}{width=0.9\textwidth, center}
        \begin{tabular}{|l|r|r|}
            \toprule
            \begin{tabular}[c]{@{}l@{}}   \textbf{Servidor OAI} → \\ \textbf{\texttt{/mnt/working/metadata}}        \end{tabular}
            & \begin{tabular}[c]{@{}l@{}} \textbf{Nº total de} \\ \textbf{metadades}                                \end{tabular}
            & \begin{tabular}[c]{@{}l@{}} \textbf{Duració de} \\ \textbf{l’abocament (minuts)}                      \end{tabular}   \\
            \midrule
            \textbf{Total}  & 242.555   &   20,95360236565 \\
            \bottomrule
        \end{tabular}
    \end{adjustbox}
    \caption{Resultat de l'abocament de les metadades a \texttt{/mnt/working/metadata}}
    \label{tab:metadata-push-filesystem}
\end{table}

\begin{table}[h!]
    \centering
    \begin{adjustbox}{width=0.9\textwidth, center}
        \begin{tabular}{|l|r|r|}
            \toprule
            \begin{tabular}[c]{@{}l@{}}   \textbf{\texttt{/mnt/working/metadata}} → \\ \textbf{MongoDB}             \end{tabular}
            & \begin{tabular}[c]{@{}l@{}} Nº total de \\ metadades                                \end{tabular}
            & \begin{tabular}[c]{@{}l@{}} Duració de \\ l’abocament (segons)                     \end{tabular}   \\
            \midrule
            \textbf{Total}  & 242.555   &   38,2050559520721 \\
            \bottomrule
        \end{tabular}
    \end{adjustbox}
    \caption{Resultat de l'abocament de les metadades a MongoDB}
    \label{tab:metadata-push-mongodb}
\end{table}
